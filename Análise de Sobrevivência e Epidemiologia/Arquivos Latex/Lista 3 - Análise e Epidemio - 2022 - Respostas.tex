\documentclass[12pt,a4paper]{article}
\usepackage[utf8]{inputenc}
\usepackage[T1]{fontenc}
\usepackage{amsmath}
\usepackage{amsfonts}
\usepackage{amssymb}
\usepackage{graphicx}
\usepackage[width=0.00cm, left=3.00cm, right=2.00cm, top=3.00cm, bottom=2.00cm]{geometry}
\begin{document}
	\begin{center}
		$(20-x)x = y$\\ 
		\vspace{0.25cm}
		$20x - x^{2} = y$\\
		\vspace{0.25cm}
		$\dfrac{d(y)}{dx} = \dfrac{20x - x^{2}}{dx}$\\
		\vspace{0.25cm}
		$f'(x) = 20 - 2x$\\
		\vspace{0.5cm}
		Para achar os pontos de máximos e mínimos locais devemos igualar a primeira derivada a zero.\\
		\vspace{0.5cm}
		$20 - 2x = 0$\\
		\vspace{0.25cm}
		$20 - 2x = 0$\\
		\vspace{0.25cm}
		$2x = 20$\\
		\vspace{0.25cm}
		$x = \dfrac{20}{2}$\\
		\vspace{0.25cm}
		$x = 10$\\
		\vspace{0.5cm}
		Determinando ponto de máximo pela concavidade ser voltada para baixo.\\
		\vspace{0.5cm}
		$(20-10)10 = y$\\
		\vspace{0.25cm}
		$10 * 10 = y$\\
		\vspace{0.25cm}
		$y = 100$
	\end{center}
\end{document}